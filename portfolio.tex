\documentclass{report}
\usepackage{csquotes}
\usepackage{tikz}
\usepackage{booktabs}
\usepackage{graphicx}
\usepackage{fullpage}
\usepackage{tabularx}
\usepackage[export]{adjustbox}
\usepackage[dutch]{babel}
\usepackage{float}
\usepackage{titlesec}
\usepackage{enumitem}
\usepackage{pdfpages}
\usepackage[backend=biber,style=apa,natbib=true]{biblatex}
\DeclareLanguageMapping{dutch}{dutch-apa}
\addbibresource{bronnen.bib}


\addto\captionsdutch{\renewcommand{\chaptername}{}}

\titleformat{\chapter}[hang]
  {\normalfont\huge\bfseries}
  {\thechapter\ }
  {0pt}
  {}

\graphicspath{{images/}}

\begin{document}

\begin{titlepage}
    \thispagestyle{empty}
    \begin{center}
        \includegraphics[width=1\textwidth]{hogeschool}\\[2cm]
        {\Huge \textbf{Afstudeerportfolio}}\\[0.3cm]
    \end{center}

    \vfill

    \begin{flushleft}
        \setlength{\tabcolsep}{12pt}
        \begin{tabular}{@{} l l @{}}
            \textbf{Auteur}& Xander Tamis\\[0.25cm]
            \textbf{Organisatie}& Breinstein\\[0.25cm]
            \textbf{Begeleiders}\\
            & Jorrit de Haas, Hogeschool van Amsterdam\\
            & Ivo Weel, Breinstein\\[1cm]
        \end{tabular}
    \end{flushleft}
\end{titlepage}

\newpage

\begin{figure}[H]
    \includegraphics[width=0.4\textwidth]{breinstein}
\end{figure}
\space

\vfill

\section*{Afstudeerportfolio}
Auteur: Xander Tamis
\\
Studentnummer: 500843085
\\
\\
Plaats en datum: Alkmaar, 18 november 2025
\\
\\
Onderwijsinstelling: Hogeschool van Amsterdam
\\
Opleiding en leerroute: HBO-ICT, Software Engineering
\\
Begeleidende docent: Jorrit de Haas
\\
\\
Organisatie: Breinstein
\\
Adres: Comeniusstraat 8, 1817 MS
\\
Telefoonnummer: 085 066 7421
\\
Bedrijfsbegeleider: Ivo Weel
\\
Stageperiode: Semester 1 2025/2026
\newpage

\tableofcontents

\newpage

\chapter{Actueel CV}

Dit hoofdstuk bevat mijn meest recente en volledig bijgewerkte curriculum vitae, 
aangepast op \textbf{18 november 2025}. Het CV geeft een overzicht van mijn professionele en 
persoonlijke ontwikkeling en vormt daarmee een belangrijk onderdeel van dit portfolio.
\\\\
Mijn CV bevat onder meer:
\begin{itemize}
    \item een overzicht van mijn werkervaring, inclusief relevante functies, projecten en verantwoordelijkheden;
    \item een opsomming van mijn opleidingen en behaalde diploma’s;
    \item een beschrijving van mijn vaardigheden, zowel technisch als persoonlijk;
    \item mijn stage-ervaringen, inclusief beschrijvingen van de organisaties;
    \item mijn persoonlijke informatie en contactgegevens;
    \item een korte toelichting op mijn hobby’s en interesses die bijdragen.
\end{itemize}
Het doel van het opnemen van dit CV is om de context van mijn leer- en werkervaringen inzichtelijk te maken, 
en om te laten zien hoe deze ervaringen bijdragen aan mijn professionele groei. Het vormt samen met de overige onderdelen van dit portfolio een 
compleet beeld van mijn ontwikkeling, kwaliteiten en ambities.

\includepdf[pages=-]{CV Xander Nederlands.pdf}

\newpage

\chapter{Leervermogen}

In dit hoofdstuk wordt mijn ontwikkeling met betrekking tot de competentie leervermogen toegelicht. 
\\\\
Aan de hand van een uitgewerkt STARR-formulier beschrijf ik een situatie waarin mijn leerproces centraal stond en waarin duidelijk wordt hoe ik mijn eigen manier van leren heb toegepast en verder heb ontwikkeld. 
\\\\
Vervolgens wordt in het Competentie Ontwikkelplan gereflecteerd op mijn groei, mijn leerstrategie en de vaardigheden die ik in toekomstige projecten verder wil versterken.

\section{STARR}

\subsection{Situatie}

\subsection{Taak}

\subsection{Actie}

\subsection{Resultaat}

\subsection{Reflectie}

\section{Competentie Ontwikkelplan}

\chapter{Beroepsethiek en maatschappelijke oriëntatie}

\section{STARR}

\subsection{Situatie}

\subsection{Taak}

\subsection{Actie}

\subsection{Resultaat}

\subsection{Reflectie}

\chapter{Samenwerken}

\section{STARR 1}

\subsection{Situatie}

\subsection{Taak}

\subsection{Actie}

\subsection{Resultaat}

\subsection{Reflectie}

\section{STARR 2}

\subsection{Situatie}

\subsection{Taak}

\subsection{Actie}

\subsection{Resultaat}

\subsection{Reflectie}

\section{Feedbackformulier 1}

\section{Feedbackformulier 2}

\section{Feedbackformulier 3}

\end{document}